\documentclass{article}
\usepackage[T1]{fontenc}
\usepackage[a4paper, landscape]{geometry}
\usepackage{hyperref}
\usepackage{nopageno}
\usepackage{parskip}
\usepackage{luatexja-fontspec}
\setmainjfont{IPAMincho}

\begin{document}
I visited Naoshima towards the end of the Japanese Winter in 2024. 南寺
(minamidera) left an impression in my mind. This place was one of the
places I've been to that motivated me to start writing music again.

\section*{Structure}
This piece explores the idea of musical parameters as states, and the
transition of the states, in a manner perhaps similar to Iannis Xenakis'
Analogique A et B. There are four sections, with each of them starting
from a list of predetermined initial states. The initial states are then
transformed into subsequent states with a Markov process.

\section*{Performance Instruction}
The piece can be performed as it is, for solo piano, fully notated.
%
It can also be performed as a mix of playing the notated music and
improvising.
%
In this case, the notated music can serve as a guide for the
improvisation.
%
Metric modulation indicates the transition from one section to the next.
%
When improvising, this piece can be played accompanied.

\noindent
Accidentals apply only to the note they are attached to.

\noindent
Dynamics are notated above the staff rather than next to the notes.
\end{document}
